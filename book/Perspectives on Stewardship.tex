\documentclass[11pt,a5paper]{article}
\usepackage[utf8]{inputenc}
\usepackage{amsmath}
\usepackage{amsfonts}
\usepackage{amssymb}
\usepackage{graphicx}
\usepackage[super]{nth}

\title{Perspectives on Stewardship}
\author{Marcel Gietzmann-Sanders}
\date{}
\setcounter{tocdepth}{1}
\begin{document}
\maketitle
\tableofcontents
\newpage

\section{What it Means to Manage}

"Taking a close look at what's around us, there is some sort of a harmony. It is the harmony of overwhelming and collective murder." - Werner Herzog on the Amazon rainforest, 1981. \newline

I absolutely love this quote. Not because I dislike rainforests or am a sadist but because it is illuminating an essential part of the natural world that we rarely think about. If I bring up rainforests, the arctic, or the oceans, to most people that is synonymous with beautiful butterflies enjoying the sunlight filtering down through a magnificent canopy, polar bear cubs playing in the snow, or reefs teeming with life. If ever something dark and dismal is presented it is us tearing down trees, melting the bears' home, or bleaching the reefs. We are the violence and we must stop it. 

However, in truth the natural world is a massive struggle for survival that takes no prisoners. Bears eat the seals that look so cute to us. Trees try to crowd out the light that would feed the children of their competitors. Insects go about parasitizing each other and many other creatures. It is a massive race to get before you're gotten. The harmony is not one of collective, atruistic cooperation but of struggle. 

Yet it is precisely this struggle that creates the very sustainability and stability that we depend upon and seek to emulate. Plants take up nutrients and energy which becomes food for hungry grazers who in turn become the prey of other creatures or disease and so it goes until all is returned to the earth where scavengers and eventually microorganisms return those nutrients back to the soil so that the process can start anew. The cycle of life and death is the cycle of renewal. 

Likewise stability is a result of these interplays, specifically in the form of negative feedback loops. Predators, food availability, and disease regulates the abundance of prey while the abundance of disease and prey availability regulates the abundance of predators. A gross over simplification to be sure, but it gets the point across - nature regulates itself through feedback. \newline

Nature then is not in harmony with itself because of some kind of stasis or inherent permanence but instead because hundreds of thousands of different dynamic elements are trying to gain ground and through their collective competition create the cycles that allow for sustainability and the negative feedback loops that allow for stability. Nature is the sustainability engine it is through its dynamic non-linearity. \newline

With this in mind I believe the place of management becomes clear. But let me illustrate with an example. Waste from human processes (better known today as pollution) has become a problem as of late. Run off enters our streams, estuaries, and oceans and has been known to create devastating blooms of algae that choke out the local life. What has happened here? Quite simply our interaction with the feedback loops present sent them cascading to a new "stable" point. When the waste showed up in the quantities it did microorganisms, who are known for their exceptional ability to multiply, were able to increase in quantity at rates that far exceeded any of their predators ability to compensate. The feedback loop could not keep pace. Eventually the mass of microorganisms got so large that they starved the area of much needed oxygen and their would be regulators died off instead.  

Consider instead what would have happened if that waste had more slowly leached into the system. If the rate was slow enough then these blooms would not have happened. It is not that the waste ended up there that is the immediate problem it is the fact that it came in so fast that it completely overwhelmed the feedback loop. 

We human beings \textit{will} produce waste and we \textit{will} consume resources from the earth. There is no such thing as a humanity separate from its biosphere. We are as much a part of the wild as anything else. The question to us is not whether we will engage with and shift the world by our presence (as compared to our absence) but rather whether the way we engineer our interactions results in \textit{dynamic} stability or not. Do we engineer interactions that mesh well with the feedback cycles out there or do we topple them over and race towards a great unknown? \newline 

\textit{Good management means engineering interactions with our dynamic, non-linear biosphere that take advantage of the cycles and feedback loops present to achieve dynamic stability and sustainability. Change is inevitable, instability is not.} \newline 


In order for us to be good managers then, we need to understand the feedback loops and cycles that are present. And this means looking at the world around us and asking - "what jobs do you do?" For example much of the issues that have arisen with industrial farming have come from not appreciating all the kinds of value the biosphere has been/was bringing us in the first place \cite{biomimicry}. Soil erosion, nitrogen fixing, pest mitigation, weather resistance, all of these are examples of "jobs" we've realized the biosphere has already figured out only after problems have cropped up from our having eliminated the species doing those jobs. Soil erosion is preventable if you have plants that can cover the ground and slow the water as it hits the earth as rain. Not as much nitrogen fertilizer is needed if nitrogen fixers are planted in fields as well whatever else is being grown there. Diversity in crops helps prevent plagues and just generally ensures not all of our eggs have been placed in one basket.

Likewise following the chain of feedback loops and "jobs" performed helps ensure we don't insert ourselves into an ecosystem attempting to achieve one kind of value while accidentally destroying another. Catching one kind of fish in great quantities that is prey for another valuable kind of fish can result in amping up one fishery while decimating another. Clearing predators in order to bump up hunting can eventually just result in poorer and poorer quality game as the predators are no longer around to put adaption pressure on the populations that are left. Removing wetlands in order to get more farmland and simply removing nurseries for valuable fish species. All in all it is important to recognize that not only are there many "jobs" being done but there are usually many values being derived from ecosystems simultaneously and unless we recognize these we are bound to erase things we depend upon by accident. \newline

\textit{Good management means understanding the full breadth of these systems, both in terms of all of the "jobs" performed to keep the system stable and sustainable, but also in terms of the various "jobs" performed that bring direct human value. That which goes unnoticed ends up being that which is eventually lost. }\newline

However this requirement for breadth presents us with a rather big problem. There are quite literally millions of different species distributed across our planet and that means there are an absolutely massive number of interactions, cycles, and feedback loops. Indeed this number gets even more severe if you think about the fact that each of these kinds of interactions play out in many different places at many different times across the globe. How can we possibly know all of this? Naively, if we were trying to describe each of these interactions and processes in hyper-specific detail we know, for a fact, that the task would be impossible - such knowledge would be more or less equivalent to omniscience. However we probably do not need such painfully precise detail. 

To illustrate, consider your car (or someone else's if you do not have one). Cars these days are extremely complicated. There are loads of different systems operating at all times when you are driving one. Pumps, engine(s), coolant systems, hydraulics, differentials, navigation systems, stereos, safety systems, lights... the list goes on and on. Yet you do not have to know all of these in all of their detail to drive the car around safely. Indeed driving a car is relatively easy. Building a car from scratch, however, takes the industry of many, many souls. The key is that you know what things you \textit{need} to pay attention to and what things you can leave as a black box. You know you need to steer with the steering wheel, use the gas and break pedals, indicate, and in certain more specific cases use the windshield wipers, understand that the oil light means its time to go to the mechanic, and so on. The operating abstraction of the car means that you needn't know how every last component in your car works to operate it. 

In a similar fashion we need an operating abstraction of ecosystems. We need to know what kinds of knowledge we need to get our hands on to operate responsibly, but also what information can be left a mystery so that the problem become tractable at all. We need to know how to break down stability and sustainability in a system to its basic, abstract parts. Thankfully such a field exists! Or rather, such fields exist and they are the fields of cybernetics, systems thinking/design, chaos theory, and stability theory. These four deal with how to understand and abstract non-linear systems and their dynamics. It will be (and to some extent already is) in applying and advancing these fields that a theory of ecological management arises. \newline

\textit{Good management takes advantage of the fields of cybernetics, systems thinking/design, chaos theory, and stability theory to provide operating abstractions of ecosystems that makes the problem of management tractable.}\newline

In sum then management is about taking the mathematical fields outlined above to produce an operating abstraction of ecological stability and sustainability, doing the research to describe the various "job" and values in the terms outlined by that operating abstraction, and then designing the human engagement in that system to achieve those values with dynamic stability and sustainability. Theory to knowledge to practice. \newline

Our next step then in understanding what it means to manage is to find that operating abstraction that will allow us to avoid the proverbial boiling of the ocean. 





\bibliographystyle{plain}
\bibliography{reference}
\end{document}